%!TEX output_directory = .aux
%!TEX copy_output_on_build(true)

\documentclass[11pt,a4paper, titlepage]{article}
\usepackage[a4paper, total={6.5in, 8in}]{geometry}
\usepackage[utf8]{inputenc}
\usepackage{amsfonts}
\usepackage{amssymb}
\usepackage{amsmath}
\usepackage{mathtools}
\usepackage{amsthm}

\title{A Study of the Function and Structure of Algorithms and Data Structures}
\author{Giannis Tyrovolas}
\date{February 8, 2020}
\newtheorem{theorem}{Theorem}[section]
\newtheorem{prop}[theorem]{Proposition}
\DeclarePairedDelimiter\abs{\lvert}{\rvert}
\DeclarePairedDelimiter\norm{\lVert}{\rVert}

\theoremstyle{definition}
\newtheorem{definition}[theorem]{Definition}
\newtheorem{example}[theorem]{Example}
\newtheorem{corollary}[theorem]{Corollary}
\newtheorem{lemma}[theorem]{Lemma}
\newtheorem{proposition}[theorem]{Proposition}
\newtheorem{algo}[theorem]{Algorithm}

\begin{document}

\maketitle

\section{Lecture 5: Network Flow I}

Imagine flow networks as a set of pipes with water going through them. Since there's a direction to the water the graph is obviously a directed graph. Each edge is basically a tube. Each tube has a certain \emph{integer} capacity and we think of start and finishing points. There's a lot of problems you can think of but an obvious question is what's the maximum amount of water you can send from the \emph{source} to the \emph{sink}. 
\\
Although not necessary, for our sake we'll consider flows with the following properties:
\begin{itemize}
	\item no anti-parallel edges
	\item no edge enters $s$
	\item no edge leaves $t$
\end{itemize}

Now to write all this in a fancy way:

\begin{definition}[Flow Networks]
A \emph{flow network} is a tuple $(G, s, t, c)$ where $G = (V,E)$ is a directed graph, $s, t \in V$ and $c\colon E \longrightarrow \mathbb{N}$.
\\
The other conditions can be stated as:
\begin{itemize}
	\item $(u,v) \in E \implies (v,u) \notin E$
	\item $\nexists u$ such that $(u,s) \in E$
	\item $\nexists v$ such that $(t,v) \in E$
\end{itemize}
\end{definition}

We also need to define a particular flow:

\begin{definition}[Flow]
A \emph{flow} is an assignment of load to each edge of the form $f\colon E \longrightarrow \mathbb{R}_{\geq0}$ That respects the capacity of each edge, i.e.
\[
	\forall e \in E \ 0 \leqslant f(e) \leqslant c(e)
\]
Also much load enters a node leaves the node. That is except for $s$ and $t$ So:
\[
	\forall v \in V \sum_{e \text{ into } v} f(e) = \sum_{e \text{ leaving } v} {f(e)}
\]
\emph{Value} of $f$, denoted $\abs{f}$ is the sum of the load of the edges leaving $s$.
\end{definition}

Now, lets think of how we can maximise $\abs{f}$. Before finding a solution lets think of an upper bound. 
\\
\\
Say we partition $V$ in $A$ and $B$ with $s \in A$ and $t \in B$. Then the maximum flow from $s$ to $t$ cannot exceed the total amount of flow leaving $A$. If we call this partition an $s$-$t$ cut and $cap(A,B) \coloneqq \displaystyle \sum_{e \text{ leaves } A} c(e)$. We derive the following:

\begin{lemma}
For any $s$-$t$ cut: $\abs{f} \leqslant cap(A,B)$. 
\end{lemma}

Hence, finding the minimum cut gives a potentially good bound on the maximum flow. But this has an easy corollary.

\begin{corollary}
For a partition $(A,B)$ and a flow $f$: 
\[
\abs{f} = cap(A,B) \implies f \text{ is a maximum flow and } (A,B) \text{ is a minimum cut.}	
\]
\end{corollary}
\begin{proof}
For a set $S$ and a set of upper bounds $B$ if $s\in S, \ b \in B$ and $s = b$ then $s$ is maximum and $b$ is minimum. 
\end{proof}

Now for an actual solution. The first thing all computer scientists should try is a greedy algorithm. Of course this fails in this case.

\begin{algo}[Greedy max-flow]
Pick any path from $s$ to $t$ and assign the maximum load possible. Repeat until no improvement can be made.
\end{algo}

This won't work as an arbitrary assignment is inefficient. There might be multiple ways of going from node $u$ to $t$ but an arbitrary choice might "fill" a unique path from $v$ to $t$.
\\
Now there's the Ford-Flukerson algorithm which is the greedy algorithm that works. It works because if you can ``fill up'' a node.
The idea is that you select a path, add as much as you can and then add reverse edges for each edge in that path. 

\end{document}

